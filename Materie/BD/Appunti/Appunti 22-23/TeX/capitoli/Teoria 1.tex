\chapter{Teoria 1}

\section{Introduzione}

Le \textcolor{blue}{basi di dati} sono un insieme di \textcolor{blue}{dati} utilizzati per il supporto allo svolgimento di attività.

Per esempio, in ambito universitario, bisogna tenere memorizzate grandi quantità di dati relativi a ogni studente. Per fare ciò non si possono usare array, alberi, liste, etc. perchè la memoria principale non può gestire velocemente tanti dati. 

All'interno delle basi di dati i dati sono \textcolor{blue}{integri, flessibili} e con una \textcolor{blue}{ridondanza controllata}, ma hanno costantemente bisogno di \textcolor{blue}{manutenzione}.

\subsection{Sistemi informativi}

\label{Sistemi informativi}

I \textcolor{blue}{sistemi informativi} gestiscono le informazioni di interesse. Ogni organizzazione ha un sistema informativo, esplicito o meno ed è indipendente dall'automazione (es. I banchieri fiorentini nel '500). Esso si occupa di:
\begin{itemize}
    \item raccolta e acquisizione dei dati;
    \item archiviazione e conservazione;
    \item elaborazione, trasformazione e produzione;
    \item distribuzione, comunicazione e scambio.
\end{itemize}

Un \textcolor{blue}{dato} è qualcosa di immediatamente percepibile, mentre un'\textcolor{blue}{informazione} è costituita da dati uniti tramite delle interpretazioni. 

\section{DBMS}

Le basi di dati devono essere gestite da un \textcolor{blue}{DBMS}\footnote{\textit{Database management system}} che è esterno alle applicazioni. Alcuni esempi sono MySQL\footnote{Inizialmente libero, ma poi acquisito da Oracle}, MariaDB\footnote{Versione opensource di MySQL} e PostgreSQL. I DBMS sono persistenti poichè la loro vita è indipendente dalle applicazioni che li utilizzano, sono condivisibili in modo da ridurre eventuali ridondanze (informazioni ripetute) e incoerenze (allineamento scorretto dei dati). Tuttavia la condivisibilità delle basi di dati produce concorrenza (accesso contemporaneo a stessi dati) che deve essere regolamentata al fine di salvaguardare l'integrità. I DBMS garantiscono anche la privatezza mediante vari tipi di autorizzazione. 

Inoltre sono affidabili, ovvero resistenti a malfunzionamenti, tramite la \textcolor{blue}{gestione delle transazioni}. Una transazione\footnote{vedi \ref{Transazioni}} è un'operazione atomica (indivisibile) i cui risultati sono permanenti. Se si verifica un errore \textit{durante} lo svolgimento di una transazione i risultati vengono anullati.

\section{Modello di dati}

Un \textcolor{blue}{modello di dati} serve a organizzare e descrivere i dati. In ogni base di dati esistono:
\begin{itemize}
    \item schema: descrive la struttura;
    \item istanza: i valori effettivi in un determinato istante.
\end{itemize}

Un \textcolor{blue}{modello concettuale} serve per progettare una base di dati a un livello astratto. Di solito si usa il modello \textcolor{blue}{\textit{entity-relationship (ER)}}\footnote{\textit{Modello entità-associazione}}.

I \textcolor{blue}{modelli logici} sono utilizzati dai programmi. In questo corso useremo il modello \textcolor{blue}{\textit{relazionale}}.
Per rappresentare il modello logico si usano:
\begin{itemize}
    \item schema logico: la descrizione base di dati nel modello logico;
    \item schema esterno: descrizione di parte della base di dati in un modello logico;
    \item schema interno (o fisico): rappresenta lo schema logico tramite strutture di memorizzazione (es. record di puntatori).
\end{itemize}

Il livello logico è indipendente da quello fisico. In questo corso si vedrà solo il livello logico.

L'accesso ai dati avviene solo tramite il livello esterno (che può coincidere con quello logico), generando due forme di indipendenza:
\begin{itemize}
    \item logica: Il livello esterno è indipendente da quello logico;
    \item fisica: Il livello logico e quello esterno sono indipendenti da quello fisico. 
\end{itemize}

\section{Linguaggi per basi di dati}

Si hanno diversi linguaggi:
\begin{itemize}
    \item \textcolor{blue}{interfacce grafiche}: senza un linguaggio testuale;
    \item \textcolor{blue}{SQL}: linguaggio testuale interattivo;
    \item \textcolor{blue}{comandi SQL}: si possono usare in vari linguaggi;
    \item \textcolor{blue}{SQL in linguaggi ad hoc}: linguaggi propri di un DBMS.
\end{itemize}

Oltre a questo i linguaggi si dividono in due macrocategorie: 
\begin{itemize}
    \item \textcolor{blue}{DML}\footnote{\textit{Data manipulation language}}: servono per interrogare e aggiornare le istanze del DBMS;
    \item \textcolor{blue}{DDL}\footnote{\textit{Data definition language}}: definiscono schemi ed eseguono operazioni generali. 
\end{itemize}

\section{Persone che interagiscono con un DBMS}

Breve elenco di alcune categorie che interagiscono con i DBMS:

\begin{itemize}
    \item Progettisti e realizzatori di DBMS;
    \item DBA\footnote{Amministratori della base di dati};
    \item Progettisti della base di dati;
    \item Progettisti e programmatori di applicazioni;
    \item Utenti finali: categoria di utenti che eseguono transazioni;
    \item Utenti casuali: categoria di utenti che eseguono operazioni diverse da transazioni.
\end{itemize}