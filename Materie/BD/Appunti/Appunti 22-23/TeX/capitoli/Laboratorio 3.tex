\chapter{Laboratorio 3}

\section{Progettazione concettuale}

L'analisi inizia con i primi requisiti raccolti (in linguaggio naturale).
Le possibili fonti sono:
\begin{itemize}
    \item utenti, attraverso documenti o interviste;
    \item documentazione già esistente, come normative o regolamenti interni;
    \item realizzazioni preesistenti.
\end{itemize}

\subsection{Acquisizione tramite interviste}
Utenti diversi possono fornire informazioni diverse (complementari o contradditorie). Gli utenti ad alto livello vedono il quadro generale, mentre gli utenti a basso livello vedono i dettagli.

\subsection{Suggerimenti per la progettazione}

Se un concetto ha proprietà significative e descrive oggetti con esistenza autonoma è un'entità.

Se un concetto è semplice e non ha proprietà è un attributo.

Se un concetto lega tra loro due o più concetti è un'associazione.

Se un concetto è un caso particolare di un altro concetto è una generalizzazione.

\subsection{Requisiti (documentazione descrittiva)}

Si deve scegliere il corretto livello di astrazione, standardizzare la struttura delle frasi ed evitare frasi contorte. Si devono unificare i termini eliminando omonimi\footnote{Hanno lo stesso nome, ma si riferiscono a concetti diversi} e sinonimi\footnote{Hanno nomi diversi, ma si riferiscono allo stesso concetto}, rendendo espliciti i riferimenti tra i termini. Si deve costruire un \textcolor{blue}{glossario dei termini} (tabella).

\section{Pattern di progettazione}

Sono soluzioni progettuali pronte per problemi comuni\footnote{Per la spiegazione in dettaglio si rimanda alle slide}:

\begin{itemize}
    \item Reificazione di attributo di entità: è la trasformazione di un attributo in un'identità;
    \item Part-of: due casi, il primo nel quale una parte non può esistere senza l'intero e il secondo in cui la parte può esistere senza l'intero;
    \item Instance-of: rappresenta il concetto istanza-classe;
    \item Reificazione di un'associazione binaria: si trasforma l'associazione binaria in un'entità;
    \item Reificazione di un'associazione ricorsiva: si trasforma l'associazione ricorsiva in un'entità;
    \item Reificazione di associazione ternaria: si trasforma l'associazione ternaria in un'entità;
    \item Reificazione di attributo di associazione;
    \item Caso particolare di un'entità: livelli diversi della generalizzazione partecipano ad associazioni diverse;
    \item Storicizzazione di un’entità: si usa la generalizzazione per rappresentare le informazioni correnti e contemporaneamente tenere traccia dello storico;
    \item Storicizzazione di un’associazione;
    \item Evoluzione di un concetto: si usa la generalizzazione per rappresentare l’evoluzione di un concetto mettendo nel genitore gli attributi e le associazioni comuni.
\end{itemize}

\section{Strategie di progetto}

\paragraph{Top-Down:} si individuano i concetti più importanti e si procede per raffinamenti successivi. Essa è conveniente perchè permette di trascurare momentaneamente alcuni dettagli, ma la si può utilizzare solo quando si ha una visione generale del progetto.

\paragraph{Bottom-Up:} le specifiche vengono divise in parti più semplici e poi unite alla fine. Questa strategia è adatta per i progetti di gruppo, ma l'integrazioni di varie parti può essere difficoltosa.

\paragraph{Inside-Out:} è una variante della Bottom-Up in cui si parte dai concetti più importanti e ci si espande a macchia d'olio. Non richiede integrazione, ma è necessario rivisitare periodicamente i requisiti per essere certi di rappresentare tutti i concetti.

\paragraph{Mista:} nella realtà si procede con una soluzione ibrida.

\section{Qualità di uno schema concettuale}

\paragraph{Correttezza:} devono essere utilizzati propriamente i costrutti messi a disposizione dal modello concettuale di riferimento.

\paragraph{Completezza:} deve modellare tutte le specifiche.

\paragraph{Leggibilità:} deve poter essere compreso in maniera immediata.

\paragraph{Minimalità:} le specifiche devono presentarsi una volta sola.