\chapter{Laboratorio 4}

\section{Progettazione logica}

La \textcolor{blue}{progettazione logica} è la fase successiva alla progettazione concettuale. 

Dati in ingresso: 
\begin{itemize}
    \item schema concettuale;
    \item informazioni sul carico applicativo;
    \item modello logico che si vuole adottare.
\end{itemize}

Dati in uscita: 
\begin{itemize}
    \item schema logico;
    \item vincoli di integrità;
    \item Documentazione associata.
\end{itemize}

Si può dividere in due sottofasi: la ristrutturazione dello schema concettuale (ER) e la traduzione verso il modello logico con relative ottimizzazioni.

\section{Ristrutturazione dello schema ER}

La \textcolor{blue}{ristrutturazione dello schema ER} serve a semplificare la traduzione nel modello logico e a ottimizzare le prestazioni. Si usano due indicatori per tenere traccia delle prestazioni: il tempo (numero di occorrenze visitate per un'operazione nel DB) e lo spazio (quantità di memoria per rappresentare i dati). Per poter valutare questi parametri si ha bisogno di alcune informazioni: il volume dei dati (numero di occorrenze e dimensione degli attributi) e le caratteristiche delle operazioni (se interattive o di batch, la frequenza e il numero di entità/associazioni coinvolte). Queste informazioni  vengono rappresentate su apposite tavole.

Per la ristrutturazione dello schema concettuale si possono seguire i seguenti passi.

\subsection{Analisi delle ridondanze}

Una \textcolor{blue}{ridondanza} è un'informazione significativa, ma ricavabile da altre informazioni. Si deve decidere se eliminare le ridondanze o mantenerle, con un calcolo. Le ridondanze rendono più efficienti le operazioni di interrogazione/lettura dei dati, ma rendono meno efficiente l'inserimento e la modifica dei dati, inoltre occupano spazio in memoria.

\subsection{Eliminazione delle generalizzazioni}

Le generalizzazioni non sono rappresentabili nel modello relazionale per cui vanno sostituite con entità, associazioni o regole aziendali (\textit{business rules}). In generale si hanno tre modi per eliminare una generalizzazione:

\begin{itemize}
    \item si accorpano i figli della generalizzazione nel genitore ("uccidendo" le entità figlie);
    \item si accorpa il genitore della generalizzazione nei figli ("uccidendo" l'entità genitore);
    \item si sostituisce la generalizzazione con associazioni, aggiungendo eventuali business rules.
\end{itemize}

Si possono anche adottare soluzioni ibride.

\subsection{Partizionamento di concetti}

Si separano attributi di uno stesso concetto ai quali si accede in operazioni diverse e si accorpando attributi di concetti diversi a cui si accede con le medesime operazioni. Spesso è possibile rimandare questo problema alla fase di progettazione fisica (che non è argomento di questo corso).

\subsection{Scelta degli identificatori principali}

Si devono scegliere gli identificatori che diventeranno chiave primaria seguendo alcuni criteri:

\begin{itemize}
    \item assenza di opzionalità;
    \item semplicità;
    \item utilizzo nelle operazioni più frequenti/importanti.
\end{itemize}

Se nessun identificatore rispetta questi criteri si devono introdurre nuovi attributi (per esempio codici) che serviranno da chiave primaria.

\subsection{Attributi composti e attributi multivalore}

Si possono trasformare gli attributi multivalore, reificando l’attributo e aggiungendo un’associazione. Gli attributi composti non sono rappresentabili direttamente in relazionale e devono essere trasformati.

\section{Traduzione verso il modello relazionale}

Le idee di base sono due:
\begin{itemize}
    \item le entità diventano relazioni con gli stessi attributi delle entità;
    \item Le associazioni diventano relazioni con attributi delle associazioni + gli identificatori delle entità coinvolte
\end{itemize}

Si devono dare nomi più espressivi agli attributi della chiave della relazione che rappresenta l’associazione. 

La traduzione non riesce a tener conto delle cardinalità minime delle associazioni molti a molti. La traduzione dell’associazione uno a molti riesce a rappresentare efficacemente la cardinalità minima della partecipazione che ha 1 come cardinalità massima (se è 0 ammette valori nulli, se è uno non ammette valori nulli). L’identificazione esterna è sempre su un’associazione uno a molti o un’associazione uno a uno.