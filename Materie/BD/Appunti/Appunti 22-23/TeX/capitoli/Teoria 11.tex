\chapter{Teoria 11}

\section{Transazioni}

\label{Transazioni}

In una base di dati si possono effettuare operazioni di \textcolor{blue}{inserimento} (INSERT), \textcolor{blue}{cancellazione} (DELETE) e \textcolor{blue}{aggiornamento} (UPDATE). Il DBMS accetta modifiche solo se rispettano i vincoli \footnote{vedi \ref{Vincoli}}, altrimenti le rifiuta. 

I vincoli possono bloccare istruzioni singole. La soluzione è consentire al DBMS di eseguire quelle istruzioni e attendere l'arrivo di altre istruzioni che pongano rimedio alla violazione dei vincoli. Questo ragionamento è alla base del concetto di \textcolor{blue}{transazione}. 

Una transazione è: 
\begin{itemize}
    \item un'unità di programma che inizia con \textbf{\textit{begin transaction}} che comunica al DBMS la richiesta di interazione da parte dell'applicazione;
    \item il DBMS identifica l'inizio della transazione $T_i$ e la abbina in modo univoco con l'utente/applicazionec che ne ha fatto richiesta;
    \item il DBMS riceve dei comandi DML in sequenza e li abbina alla transazione;
    \item solitamente breve (buona pratica);
    \item se la transazione va a buon fine si termina con il comando \textbf{\textit{commit work}};
    \item se la transazione fallisce si termina con il comando \textbf{\textit{rollback work}}.
\end{itemize}

\section{Ciclo di vita di una transazione}

Con stato 1 si indica la transazione attiva, con stato 2 si indica la transazione fallita e con stato 3 si indica la transazione parzialmente terminata. I due stati finali sono aborted e committed. Questa simbologia serve per una rappresentazione mediante un DFA\footnote{come visto nel corso di LFT} del ciclo di vita di una transazione.

\begin{enumerate}
    \item La transazione ha inizio (Stato 1);
    \item Finchè la transazione riceve comandi di inserimento, cancellazione e modifica rimane in questo stato (Stato 1). Se fallisce va al punto 3 (Stato 2), se va a buon fine va al punto 4 (Stato 3);
    \item Dopo il rollback la transazione può solo raggiungere lo stato finale di \textbf{aborted};
    \item Dopo il commit se la transazione rispetta i vincoli passa allo stato finale di \textbf{committed}, altrimenti passa al punto 3 (Stato 2);
\end{enumerate}

\section{Proprietà delle transazioni}

Una transazione (in una base di dati relazionale) ha le seguenti proprietà (dette ACID):

\begin{itemize}
    \item \textcolor{blue}{atomica}: una transazione avviene o per intero o non avviene affatto. La BD non può essere lasciata in uno stato intermedio. Per garantire questa proprietà si usano comandi di UNDO o REDO, più file di log;
    \item \textcolor{blue}{consistente (coerente)}: l'esecuzione di una transazione non deve violare i vincoli;
    \item \textcolor{blue}{isolata}: per garantire l'esecuzione concorrente come se ogni operazione venisse eseguita singolarmente;
    \item \textcolor{blue}{durabile (persistente)}: tutte le modifiche devono rimanere, anche se sono presenti guasti.
\end{itemize}