\chapter{Variabili aleatorie}

Una \textcolor{purple}{variabile aleatoria} è un funzione di $\Omega$ in $\mathbb{R}$.

\ex{}{Si è interessati a studiare i voti di maturità degli studenti del primo anno del CDL di informatica. Quindi si estrae uno studente (parte casuale) e lo si associa al suo voto di maturità (misurazione).}

Le variabili aleatorie possono essere:

\begin{itemize}
    \item \textcolor{purple}{discrete}: se l'immagine è numerabile.
    \item \textcolor{purple}{continue}: se l'immagine è sottoinsieme di $\mathbb{R}$
\end{itemize}

\section{Variabili aleatorie discrete}

L'esempio del paragrafo precedente è una variabile aleatoria discreta dato che il voto di maturità è espresso in centesimi, quindi con un intero.

\begin{center}
    $X(\omega) = x$
\end{center}

\begin{itemize}
    \item $X$ è la funzione che viene applicata.
    \item $\omega$ è il valore di $\Omega$.
    \item $x$ è il punto di $\mathbb{R}$
\end{itemize}

$Im(X) = \{ x \in \mathbb{R} | \exists \omega \in \Omega$ t. c. $X(\omega) = x \}$

\subsection{Probability mass-function (PMF)}

\dfn{PMF}{La \textcolor{purple}{funzione della massa di probabilità} è la funzione così definita:

$\prob_x : Im(X) = \mathbb{R}$

$x \rightarrow \prob(x) = \prob(X = x) = \prob( x \in \mathbb{R} | X(\omega) = x \}) $}

\subsection{Variabili aleatorie note}

\dfn{Bernoulli}{
Esperimento probabilistico: con esito binario, descritto da due etichette (S/F, 1/0, V/F, etc.)

Immagine: Im(x) = \{0, 1\}

PMF: $\prob_X(x) =  1 - p $ se x = 0,  $\prob_X(x) =  p $ se x = 1. $p \in \{0, 1\}$

Media:$ E(x) = p$

Varianza: $E(x) = p * (1 - p)$
}
\dfn{Binomiali}{
Esperimento probabilistico: n prove identiche ripetute.

Immagine: Im(x) = $\{(w_1, w_2, ..., w_n), w_i \in \{s, f\}\}$

PMF: $\prob_X(x) =  \binom{n}{k} * p^k * (1- p)^{n-k}$
\begin{itemize}
    \item k = 0, 1, 2, ..., n (numero di successi).
    \item p = probabilità di successo.
    \item n = numero di prove.
\end{itemize}

Media: $E(x) = n * p$

Varianza = $var(x) = n * p * (1-p)$
}
\dfn{Geometriche}{

Esperimento probabilistico: prove identiche ripetute fino a quando non si ottiene il primo successo.

Immagine: Im(x) = $\{1, 2, 3, ...\}$

PMF: $\prob_X(x) =  (1 - p)^{x - 1} * p$

Media: $E(x) = \frac{1}{p}$

Varianza: $var(x) = \frac{1 - p}{p^2}$
}
\dfn{Iper geometriche}{

Esperimento probabilistico: estrazioni senza reimbussolamento da una scatola composta da C elementi con una determinata caratteristica e N - C senza quella caratteristica.

Immagine: Im(x) = \{0, 1, 2, ..., min\{N, C\}\}

PMF: $\prob_X(x) =  \frac{\binom{C}{k} * \binom{N - C}{n - k}}{\binom{N}{n}}$

\begin{itemize}
    \item $\binom{C}{k}$ = numero di possibili estrazioni di k elementi con C;
    \item $\binom{N - C}{n - k}$ =  numero di possibili estrazioni di n-k elementi senza C;
    \item $\binom{N}{n}$ = numero di possibili estrazioni di n elementi da N.
\end{itemize}
}
\dfn{Poisson}{

Esperimento probabilistico: osservo il verificarsi di une elemento.

Immagine: Im(x) = \{0, 1, 2 ,...\}

PMF: $\prob_X(x) = \frac{\lambda^x}{x!} * e^{-\lambda}$

Media: $E(x) = \lambda$

Varianza: $var(x) = \lambda$
}
\subsubsection{Media e varianza}

\dfn{Media}{La \textcolor{purple}{media} delle variabili aleatorie discrete è la quantità: 
$$E(x) = \sum_{x \in Im(x)} x * \prob_X(x)$$
}

\cor{}{Per la legge dei grandi numeri: 
$$lim_{N\rightarrow \infty} \frac{k}{N} = \prob(A)$$
}

\dfn{Momento di ordine k}{Il \textcolor{purple}{momento di ordine k} è la seguente quantità:
$$E(x^k) = \sum_{x \in Im(x)} x^k * \prob_X(x)$$
}

\dfn{Varianza}{La \textcolor{purple}{varianza} di x è la seguente quantità:
$$var(x) = E[[x - E(x)]^2]$$
}

\begin{itemize}
    \item $[x - E]^2$ = scarto quadratico medio.
    \item $\sqrt{var(x)}$ = deviazione standard.
\end{itemize}

\paragraph{Proprietà di media e varianza}

\begin{itemize}
    \item $E(aX + bY) = a * E(X) + b * E(Y)$.
    \item $var(x) = E(x^2) - E(x)^2$.
    \item $var(aX + b) = a^2 * var(x)$
\end{itemize}

\subsection{PMF congiunte}

Le \textcolor{purple}{PMF congiunte} servono per considerare più variabili aleatorie sullo stesso esperimento.

\nt{Dalle PMF congiunte si possono dedurre le \textcolor{purple}{PMF marginali}}

\subsection{Indipendenza}

Se le variabili aleatorie x e y sono indipendenti: $E(x * y) = E(x) * E(y)$ e $var(x + y) = var (x) + var(y)$

\section{Variabili aleatorie continue}

L'immagine di x è un sottoinsieme di $\mathbb{R}$ ed è più che numerabile. Esempio: lunghezza, altezza, tempo.

\subsection{Probability density-function (PDF)}

\dfn{PDF}{La \textcolor{purple}{funzione di densità di probabilità} è così definita:

$f_x: \mathbb{R} \rightarrow \mathbb{R}^+ [0, +\infty)$

$\prob(X \in A) = \int_{a}^{b} f_x(t)dt$
}
\subsection{Variabili aleatorie note}

\dfn{Esponenziale}{
L'esponenziale è la variante continua della geometrica. Ha media $E(x) = \frac{1}{\lambda}$ e varianza $var = \frac{1}{\lambda^2}$}

\dfn{Gaussiana (normale)}{

PDF: $f(x) = \frac{1}{\sqrt{2 * \pi * \sigma}} * e^{-\frac{(x - \mu)^2}{2 * \sigma}}$

Se $\sigma$ è grande la curva è larga, se è piccolo la curva è stretta.
}
\subsection{Media e varianza}

\dfn{Media nella V. A. Continue}{
La \textcolor{purple}{media} delle variabili aleatorie continue è la quantità:
$$E(x) = \int_{-\infty}^{+\infty} x * f(x) dx$$
}

\dfn{Varianza nelle V. A. Continue}{
La \textcolor{purple}{varianza} delle variabili aleatorie continue è la quantità:
$$var(x) = \int_{-\infty}^{+\infty} (x - E(x))^2 * f(x) dx$$
}

\section{Funzione di distribuzione cumulata (CDF)}
\dfn{CDF}{
La \textcolor{purple}{funzione di distribuzione cumulata} è così definita: 
$$F(x): \mathbb{R} \rightarrow \mathbb{R}$$
Nel caso discreto:
$$x \rightarrow F_x = \sum_{k \in Im(x)} p_x(k)$$

Nel caso continuo:
$$x \rightarrow F_x = \int_{-\infty}^x f_x(t) dt$$
}