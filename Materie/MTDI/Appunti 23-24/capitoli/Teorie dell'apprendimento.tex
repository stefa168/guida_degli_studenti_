\chapter{Le teorie dell'apprendimento}

L'impostazione che si dà a un'attività didattica si basa su:

\begin{itemize}
    \item una \fancyglitter{teoria dell'apprendimento};
    \item l'idea, personale, che si ha di che cosa sia la \fancyglitter{conoscenza}.
\end{itemize}

\section{I paradigmi di apprendimento}

\dfn{Teorie dell'apprendimento}{Le \evidence{teorie dell'apprendimento} descrivono come le persone imparano. Esse sono prodotte da psicologia, pedagogia e filosofia.}

\dfn{I paradigmi dell'apprendimento}{I \evidence{paradigmi dell'apprendimento} sono classificazioni delle teorie in base ai loro tratti comuni.}
\subsubsection{}
Principalmente si individuano due macro-categorie:
\begin{itemize}
    \item istruttivismo;
    \item costruttivismo.
\end{itemize}

\section{Comportamentismo}

\dfn{Comportamentismo}{Si vuole modellare un \newfancyglitter{comportamento desiderabile}. L'apprendimento veniva valutato in base ai cambiamenti nel comportamento degli alunni. Gli studenti sono una \newfancyglitter{tabula rasa} che ricevono passivamente le informazioni per ripetizione. Si modella il comportamento tramite \newfancyglitter{rinforzi}\footnote{Punizioni corporali}.
}

\nt{Questo è un approccio istruttivista il cui focus è sulla trasmissione della conoscenza, sulla strutturazione e presentazione dei contenuti, e non sullo studente, che viene visto come un recipiente da riempire.}

\paragraph{Focus:}

\begin{itemize}
    \item Filosoficamente si ritiene vera l'esistenza di una \fancyglitter{realtà oggettiva} che può essere imparata;
    \item La conoscenza è una \fancyglitter{rappresentazione mentale} della realtà;
    \item C'è il presupposto che il mondo e la conoscenza esista anche se nessun essere umano li percepisce;
    \item Il linguaggio è il mezzo di trasporto della conoscenza.
\end{itemize}

\section{Cognitivismo}

\dfn{Cognitivismo}{Il cognitivismo supera l'idea di osservare solo i comportamenti esterni. Si elaborano alcune teorie:
\begin{itemize}
    \item il \newfancyglitter{carico cognitivo} è il carico di lavoro mentale necessario per eseguire un compito;
    \item gli \newfancyglitter{schemi} e i \newfancyglitter{modelli mentali}.
\end{itemize}
}

\nt{Lo scopo dell'educazione è quello di far ricordare e applicare la conoscenza.}

\paragraph{Focus:}

\begin{itemize}
    \item Gli studenti sono degli \fancyglitter{elaboratori di informazione};
    \item Gli insegnanti devono facilitare l'elaborazione.
\end{itemize}

\nt{Questo approccio è ambivalente: può essere sia istruttivista che costruttivista.}

\section{Costruttivismo}

\dfn{Costruttivismo}{Il costruttivismo\footnote{Ideato da Jean Piaget} si basa sullo \newfancyglitter{scetticismo}:
\begin{itemize}
    \item la conoscenza deriva dall'esperienza\footnote{Soggettiva};
    \item non c'è modo di sapere la "vera" verità.
\end{itemize}
}

\nt{
\begin{itemize}
    \item Il concetto di verità è illusorio;
    \item Non si può confrontare la rappresentazione con l'oggetto rappresentato;
    \item Si deve ricorrere alla "viabilità".
\end{itemize}
}

\dfn{Viabilità}{Una conoscenza viene definità \newfancyglitter{viabile} se ha funzionato bene nelle esperienze precedenti.}

\cor{Criteri di viabilità}{
\begin{itemize}
    \item Azioni fisiche: è viabile tutto ciò che porta allo scopo scelto;
    \item Piano concettuale: non ci deve essere contradditorietà o non coerenza logica. 
\end{itemize}
}

\ex{Alcuni metodi di apprendimento}{
\begin{itemize}
    \item Assimilazione: si incorpora un concetto in uno schema già acquisito;
    \item Accomodamento: si modifica la struttura cognitva in relazione a cose nuove.
\end{itemize}
}

\dfn{Apprendimento attivo}{L'\newfancyglitter{apprendimento attivo} è un ampio insieme di metodologie didattiche che coinvolgono gli studenti come parte \newfancyglitter{attiva} dell'apprendimento insieme all'insegnante.}

\dfn{Costruttivismo cognitivo}{Nel \newfancyglitter{costruttivismo cognitivo} lo scopo dell'educazione è quello di permettere agli studenti di creare nuova conoscenza. L'apprendimento è il processo di costruzione del signidìficato. L'insegnante ha solo lo scopo di facilitare la scoperta offrendo le risorse necessarie.}

\cor{}{Lo sviluppo cognitivo è consentito dall'ambiente culturale e l'apprendimento si deve svolgere con l'aiuto altrui. La \evidence{ZPS}\footnote{Zona di sviluppo prossimale} descrive le possibili zone di sviluppo di un bambino:
\begin{itemize}
    \item zona 1 (o zona di sviluppo attuale): lo studente può apprendere da solo;
    \item zona 2/ZPD (o zona di sviluppo prossimale): lo studente può apprendere solo se supportato;
    \item zona 3 (o zona di sviluppo potenziale): lo studente non può ancora apprendere nè da solo nè supportato. 
\end{itemize}
ZPS}

\nt{Un insegnante può agire solo sulla zona 2.}

\dfn{Socio-costruttivismo}{Nel \newfancyglitter{socio-costruttivismo} lo scopo dell'educazione è quello di permettere agli studenti di creare nuova conoscenza insieme. Si dà enfasi sulle relazioni umane.}

\subsection{Didattica costruttivista}

Le moderne teorie sono molto spesso basate sul costruttivismo:

\begin{itemize}
    \item \evidence{active learning}: le pratiche in cui gli studenti svolgono attivamente qualcosa e riflettono;
    \item \evidence{productive failure}: si chiede agli studenti di svolgere problemi mal strutturati o difficili. In seguito si forniscono le conoscenze necessarie;
    \item \evidence{problem-based learning} (PBL): problema realistico;
    \item \evidence{inquiry-based learning}: gli studenti formulano domande, raccolgono dati, li analizzano, provano a spiegarli e creano conoscenza teorica.
\end{itemize}

\nt{Il productive failure funziona una volta sola, quindi va usato come "jolly".}

\paragraph{Caratteristica della didattica costruttivista:}

\begin{itemize}
    \item[$\Rightarrow$] l'apprendimento non avviene attraverso fasi standard;
    \item[$\Rightarrow$] ogni studente deve avere la possibilità di stabilire il proprio percorso;
    \item[$\Rightarrow$] l'insegnante deve indirizzare;
    \item[$\Rightarrow$] le parole e le azioni del docente sono strumenti per apprendere;
    \item[$\Rightarrow$] si dà priorità all'esperienza diretta piuttosto che alla lezione tradizionale.
\end{itemize}

\nt{Ovviamente l'esperienza diretta va gestita dall'insegnante.}

\paragraph{Compiti del docente:}

\begin{itemize}
    \item[$\Rightarrow$] accertare le pre-concezioni degli alunni;
    \item[$\Rightarrow$] far emergere concezioni sbagliate;
    \item[$\Rightarrow$] ristabilire le idee mediante ipotesi e tentativi;
    \item[$\Rightarrow$] far elaborare una nuova interpretazione coerente a quella socialmente condivisa.
\end{itemize}

\section{Costruzionismo}

\dfn{Costruzionismo}{Il \newfancyglitter{costruzionismo} si basa sull'idea costruttivista di strutture di conoscenza. A ciò viene aggiunta l'idea che ciò accade nei contesti in cui si è coinvolti nella costruzione di un'entità (artefatto).}

\nt{Il costruzionismo è stato ideato da Seymour Papert, l'inventore di LOGO.}

\epigraph{\textbf{I ragazzi si avventurano nell'esplorazione di come loro stessi pensano. L'esperienza può essere esaltante: pensare sul pensare trasforma i ragazzi in epistemologi, un'esperienza rara anche per gli adulti.}}{\textit{Seymour Papert}}










