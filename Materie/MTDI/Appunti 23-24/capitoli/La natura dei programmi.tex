\chapter{La natura dei programmi}

\begin{itemize}
    \item [$\Rightarrow$] I programmi sono dappertutto;
    \item [$\Rightarrow$] La programmazione è sempre più presesente nell'ambito scolastico (coding).
\end{itemize}

\section{Che cos'è un programma?}

Questa domanda è molto complicata, la cui risposta cambia a seconda 
della persona a cui la si pone:
\begin{itemize}
    \item se una persona utilizza principalmente applicativi può vedere
    i programmi come strumenti per lavorare, divertirsi, ecc\dots;
    \item se una persona è un programmatore può vedere i programmi come
    una sequenza di istruzioni legata agli algoritmi.
\end{itemize}

\mlenma{}{Fornire agli insegnanti una visione ampia di
cosa sia un programma, al di là di definizioni riduttive o
stereotipate.}

\subsection{Un quadro di riferimento sulla natura dei programmi}

\paragraph{Questo \fancyglitter{framework} può essere utile per:}

\begin{itemize}
    \item capire la \newfancyglitter{centralità dei programmi} ai giorni nostri;
    \item orientare le \newfancyglitter{scelte didattiche}.
\end{itemize}

\paragraph{Il termine \fancyglitter{programma}, in senso lato, è usato:}

\begin{itemize}
    \item dai programmi che si scrivono a scuola ai sistemi operativi.
\end{itemize}

\paragraph{Si considerano solo programmi "\fancyglitter{tradizionali}":}

\begin{itemize}
    \item che implementano \newfancyglitter{algoritmi};
    \item per cui si può \newfancyglitter{spiegare il risultato} ottenuto;
    \item esclusi i programmi prodotti da macchine.
\end{itemize}

\section{Le sfaccettature dei programmi}

\paragraph{I programmi come \fancyglitter{strumenti}:}

\begin{itemize}
    \item [-] utili nel lavoro, nel tempo libero, ecc\dots;
    \item [-] vengono visti come scontati.
\end{itemize}

\paragraph{I programmi come \fancyglitter{opere dell'uomo}:}

\begin{itemize}
    \item [-] sono \textit{solitamente} prodotti con uno scopo preciso;
    \item [-] sono il risultato di scelte.
\end{itemize}

\paragraph{I programmi come \fancyglitter{oggetti fisici}:}

\begin{itemize}
    \item [-] risiedono su un mezzo fisico;
    \item [-] la loro esecuzione richiede tempo ed energia;
    \item [-] hanno bisogno di un dispositivo fisico per essere
    eseguiti.
\end{itemize}

\paragraph{I programmi come \fancyglitter{entità astratte}:}

\begin{itemize}
    \item [-] manipolano nozioni e concetti astratti;
    \item [-] elaborano simboli;
    \item [-] gli effetti fisici sono in funzione del risultato
    astratto;
    \item [-] gli algoritmi sono astratti.
\end{itemize}

\paragraph{I programmi come \fancyglitter{entità eseguibili}:}

\begin{itemize}
    \item [-] sono eseguiti da un calcolatore;
    \item [-] possono essere ripetuti;
    \item [-] l'esecutore è un agente che elabora segnali.
\end{itemize}

\paragraph{I programmi come \fancyglitter{manufatti linguistico-notazionali}:}

\begin{itemize}
    \item [-] rispettano una specifica sintassi;
    \item [-] sono un modo per esprimere idee;
    \item [-] sono un modo per comunicare idee;
    \item [-] possono essere tradotti in altre notazioni.
\end{itemize}

\nt{I programmi sono scritti da persone per essere compresi da altre
persone e per essere eseguiti da una macchina che non li comprende.}


