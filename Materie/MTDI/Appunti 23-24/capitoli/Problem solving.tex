\chapter{Problem solving}

\section{Formulare e comprendere i problemi}

\dfn{Compiti di realtà}{
Negli ultimi anni si fanno svolgere dei \newfancyglitter{compiti di realtà}, ossia dei problemi basati sulla realtà.
}

\dfn{Obblighi impliciti}{
Spesso i bambini tendono a limitarsi pensando a come ci si aspetta che loro debbano rispondere.
}

\ex{Nonna Rosa e la spesa}{

\paragraph{Testo:} Nonna Rosa vuole realizzare una macedonia alla frutta per i suoi nipotini. Le servono 2 kg di albicocche e 3 kg di pesche. Va al mercato per acquistarle. Nel banco della signora Agata (banco A) le pesche costano 1 € al kg e le albicocche 2 € al kg. Nel banco del signor Bruno (banco B) le pesche costano 2 € al kg e le albicocche 1 € al kg. 
Come è più conveniente fare l’acquisto? 
Quanto spenderà  Nonna Rosa?

\paragraph{Risposta semplice:} il procedimento che la maggior parte dei bambini è eseguire il calcolo per il banco A e il calcolo per il banco B e scegliere il banco in cui costa meno.

\paragraph{Risposta creativa:} si prendono le pesche nel banco A e le albicocche nel banco B.

\paragraph{Spiegazione:} solitamente i bambini danno una risposta semplice perchè sono stati abituati al fatto che i problemi vengono posti in un certo modo anche se nel testo non è scritto che si debba scegliere un banco.
}

\nt{È importante pensare al come enunciare i problemi. La formulazione di un problema in quanto tale deve modellare un bisogno reale\footnote{il problema di nonna Rosa insegna che con la matematica si può risparmiare}. Per progettare un compito di realtà bisogna comprendere le possibili interpretazioni dello stesso.}

\ex{Come porre i problemi}{

\paragraph{Formulazione puramente algoritmica:} problemi  che nella formulazione si riferiscono direttamente alle strutture dati e variabili che verranno usate nel programma progettato per risolverli.

\paragraph{Formulazione algoritmico-narrativa:} problemi che  nella formulazione non  si riferiscono esplicitamente alle strutture dati e variabili che verranno usate nel programma progettato per risolverli. La formulazione è inserita in una storia/narrazione. Per risolvere il problema in questo caso si deve prima interpretare il testo. 

\begin{center}
    \begin{tabular}{ | p{3,5cm} | p{4,7cm} | p{5cm} |}
        \hline
        \textbf{Compito}      
    & \textbf{Formulazione puramente algoritmica}
    & \textbf{Formulazione algoritmico-narrativa} \\ \hline\hline
    Trovare il massimo in una lista di numeri
    & Scrivere un programma/algoritmo in pseudo codice che ritorna il numero massimo in una lista di numeri
    & Si svolge una gara di salto in lungo. Trovare l'atleta che ha percorso la distanza maggiore\\ \hline
    
    Dire se un array è ordinato oppure no
    & Scrivere un programma/algoritmo in pseudo codice che ritorna vero se un dato array è ordinato falso altrimenti
    & Si ha un mazzo di carte e si vuole sapere se le carte sono messe in ordine\\ \hline
    
    Crittografare messaggi simulando un cifrario di Cesare
    & Scrivere un programma/algoritmo in pseudo codice che cambi ogni lettera con la sua successiva
    & Dei partner finanziari devono scambiarsi messaggi in codice.  I messaggi possono contenere parole, spazi e punti. Scrivere un metodo che ritorni un messaggio codificato in modo che ogni lettera sia rimpiazzata dalla successiva rispetto all’alfabeto\\
        \hline
        \end{tabular}
    \end{center}

}

\section{I problemi}

\qs{}{Che cos'è un problema?}

\paragraph{Risposta:} un problema è un'entità sconosciuta in qualche situazione. Ovviamente è necessario che ci sia qualcuno interessato a trovare una soluzione.

\nt{La motivazione, specialmente nei compiti di realtà, ricopre un ruolo fondamentale.}

\dfn{Problem solving}{Il \newfancyglitter{problem solving} è una sequenza di attività cognitive orientate a un obiettio unita alla manipolazione dello spazio del problema}

\nt{Un problema ben strutturato:
\begin{itemize}
    \item presenta tutti gli elementi necessari;
    \item richiede l'applicazione di un certo numero di regole e princìpi organizzati in un modo predittivo e prescritto;
    \item presenta soluzioni conoscibili e comprensibili.
\end{itemize}
}

\nt{Un problema non strutturato:
\begin{itemize}
    \item presenta elementi sconosciuti;
    \item ha più soluzioni o nessuna soluzione;
    \item ha più criteri di valutazione;
    \item richiede di esprimere giudizi e opinioni personali.
\end{itemize}
}

\subsection{Complessità dei problemi}

\dfn{Complessità}{
La complessità di un problema è definita da:
\begin{itemize}
    \item numero di aspetti, funzioni o variabili coinvolte;
    \item connettività tra queste proprietà;
    \item relazioni tra proprietà e stabilita nel tempo.
\end{itemize}
}

\nt{Se un problema è mal strutturato si ha una maggior complessità. I problemi ben strutturati coinvolgono un insieme vincolato di variabili prevedibili e comlessità minore.}

\nt{Si ha un problema simile nel corso di "Basi di dati" in fase di progettazione.}




