\chapter{Definizione di algoritmo}

\section{Cos'è un algoritmo?}

\nt{Nelle seguenti definizioni si fa implicitamente riferimento al solo paradigma imperativo.}

\dfn{Algoritmo}{
Un \newfancyglitter{algoritmo}: 
\begin{itemize}
    \item necessità di una condizione di terminazione $\rightarrow$ termina sempre;
    \item deve avere chiarezza e precisione nelle istruzioni $\rightarrow$ non si devono lasciare ambiguità.
\end{itemize}
}

\cor{Proprietà di un algoritmo}{
\begin{itemize}
    \item Finitezza: termina in un numero finito di passi;
    \item Precisione: ogni passo è precisamente definito;
    \item I/O: un algoritmo ha 0+ input e ha 1+ output;
    \item Fattibilità: un algoritmo deve essere effettivamente eseguibile;
    \item Correttezza;
    \item Efficienza.
\end{itemize}
}

\nt{In una scuola secondaria bisogna utilizzare un linguaggio più semplice rispetto ai termini accademici. Inoltre per definire un algoritmo o uno pseudo-algoritmo è utile immaginare un'interprete meccanico che deve avere istruzioni per ogni caso possibile.}

\dfn{Problemi computazionali}{Un \newfancyglitter{problema computazionale} è una collezione di domande, le istanze, per cui si sia stabilito un criterio astratto per riconoscere le risposte corrette.}

\ex{Massimo comun divisore}{
Ingressi:
\begin{itemize}
    \item Coppie di interi $a$, $b$ che non siano entrambi nulli. 
\end{itemize}
Uscite:
\begin{itemize}
    \item Un intero $c$ tale che:
    \begin{itemize}
        \item $c$ divide sia $a$ che $b$;
        \item se $d$ divide $a$ e $b$ allora $d$ divide $c$.
    \end{itemize}
\end{itemize}
}